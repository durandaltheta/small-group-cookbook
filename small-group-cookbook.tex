\documentclass{article}
\usepackage{framed}
\usepackage{fancyhdr}
\usepackage{lipsum}% just to generate text for the example
\pagestyle{fancy}
\fancyhf{}
\fancyhead[C]{\leftmark}
\fancyhead[R]{\thepage}
\renewcommand{\headrulewidth}{0pt}

\title{Dennis Family Cookbook \\ \large 2018 Edition}
\date{}
\begin{document}
\maketitle 

\begin{center}
    \section*{}
    \paragraph 
        “All true friendliness begins with fire and food and drink and the recognition of rain or frost. ...Each human soul has in a sense to enact for itself the gigantic humility of the Incarnation. Every man must descend into the flesh to meet mankind.” 
    \paragraph 
        - G. K. Chesterton
\end{center}

\newpage
\tableofcontents
\newpage

% =====================================================
\vspace*{\fill}
\begin{center}
    \section{Breakfasts}
\end{center}
\vspace*{\fill}
\newpage 

% -----------------------------------------------------
\subsection{Cowboy Coffee Cake [Christmas]} 
\noindent\rule[0.5ex]{\linewidth}{1pt}

%Ingredients
\begin{framed}
    \begin{itemize}
        \item 1 tbsp vinegar 
        \item ~1 cup milk
        \item 2 1/2 cups flour
        \item 2 cups brown sugar 
        \item 1/2 teaspoon salt 
        \item 2/3 cups shortening
        \item 2 tsp baking powder
        \item 1/2 tsp baking soda 
        \item 1/2 tsp cinnamon  
        \item 1/2 tsp nutmeg 
        \item 2 well beaten Eggs
    \end{itemize}
\end{framed}

%Instructions
\begin{enumerate}
    \item 
        Create sour milk by putting vinegar in an empty 1 cup measuring cup. Fill remaining space in the cup with with milk.
    \item 
        Combine flour, brown sugar, salt, and shortening till crumbly. Reserve 1/2 cup of mixture to crumble over batter later.
    \item 
        Add baking power, baking soda, cinnamon and nutmeg remaining batter. Mix thoroughly.
    \item 
        Add sour milk and well beaten eggs. Mix well.
    \item 
        Line 2 8x8 square pans with wax paper. Pour batter into pans. Sprinkle with reserved crumbs.
    \item 
        Bake at 375 degrees for approx. 25 minutes.
\end{enumerate}
\newpage

% -----------------------------------------------------
\subsection{Favorite Pancakes} 
\noindent\rule[0.5ex]{\linewidth}{1pt}

%Ingredients
\begin{framed}
    \begin{itemize}
        \item 2 cups flour (500 ml)
        \item 2 tablespoons sugar, optional (30 ml)
        \item 4 teaspoons baking powder (20 ml)
        \item 1 teapsoon salt, optional (5 ml)
        \item 2 eggs, beaten
        \item 1 1/2 cups milk (370 ml)
        \item 1/4 cup oil or melted shortening (60 ml)
    \end{itemize}
\end{framed}

%Instructions
\begin{enumerate}
    \item 
        Combine in a medium bowl the flower, sugar, baking powder and salt.
    \item 
        Combine seperately and then add to the main bowl the eggs and milk oil/shortening.
    \item 
        Stir quickly until blended. Do not beat. Cook on a hot, greased griddle, turning when bubly. Yields about 15 3-inch pancakes (8 cm)
\end{enumerate}

\paragraph
(Note: use any mixture of flours: whole wheat, white, oatmeal, rye, wheat germ, cornmeal, rice flour, millet, etc.)

% -----------------------------------------------------
\subsubsection{Apple Spice Pancakes} 
\noindent\rule[0.5ex]{\linewidth}{0.5pt}

%Ingredients
\begin{framed}
    \begin{itemize}
        \item 1 cup grated apple (250 ml)
        \item 1 tablespoon lemon juice (15 ml)
        \item 2 tablespoons sugar (30 ml) 
        \item 1/2 teaspoon cinnamon (2 ml)
    \end{itemize}
\end{framed}

%Instructions
\begin{enumerate}
    \item 
        Add the apple, lemon juice, sugar, and cinnamon to base mixture before cooking on griddle.
\end{enumerate}

% -----------------------------------------------------
\subsubsection{Banana Pancakes} 
\noindent\rule[0.5ex]{\linewidth}{0.5pt}

%Ingredients
\begin{framed}
    \begin{itemize}
        \item 3/4 to 1 cup liquid mashed ripe bananas (180-250 ml)
        \item 1 tablespoon lemon juice (15 ml) 
        \item 2 tablespoons sugar (30 ml)
    \end{itemize}
\end{framed}

%Instructions
\begin{enumerate}
    \item 
        Add the bananas, lemon juice, and sugar on to base mixture before cooking on griddle.
\end{enumerate}

% -----------------------------------------------------
\subsubsection{Cheese and Bacon Pancakes} 
\noindent\rule[0.5ex]{\linewidth}{0.5pt}

%Ingredients
\begin{framed}
    \begin{itemize}
        \item 1/2 cup grated cheese 
        \item 1/2 cup crisp crumbled bacon
    \end{itemize}
\end{framed}

%Instructions
\begin{enumerate}
    \item 
        Add the cheese and bacon on to base mixture before cooking on griddle.
\end{enumerate}

% -----------------------------------------------------
\subsubsection{Ham Pancakes} 
\noindent\rule[0.5ex]{\linewidth}{0.5pt}

%Ingredients
\begin{framed}
    \begin{itemize}
        \item 1/2 to 1 cup ground or chopped ham (120-250 ml)
    \end{itemize}
\end{framed}

%Instructions
\begin{enumerate}
    \item 
        Add the ham on to base mixture before cooking on griddle.
\end{enumerate}

% -----------------------------------------------------
\subsubsection{Pineapple Pancakes} 
\noindent\rule[0.5ex]{\linewidth}{0.5pt}

%Ingredients
\begin{framed}
    \begin{itemize}
        \item 1 cup pineapple
        \item 3/4 cup pineapple juice
        \item 3/4 cup powdered milk
    \end{itemize}
\end{framed}

%Instructions
\begin{enumerate}
    \item 
        Use the pineapple juice and powedered milk *INSTEAD* of regular milk in the base mixture. Add the pineapple on to base mixture before cooking on griddle.
\end{enumerate}

% -----------------------------------------------------
\subsubsection{Raisin Pancakes} 
\noindent\rule[0.5ex]{\linewidth}{0.5pt}

%Ingredients
\begin{framed}
    \begin{itemize}
        \item 1 cup raisins (250 ml). 
    \end{itemize}
\end{framed}

%Instructions
\begin{enumerate}
    \item 
        Add the raisins on to base mixture before cooking on griddle. Serve with jam.
\end{enumerate}

% -----------------------------------------------------
\subsubsection{Waffles} 
\noindent\rule[0.5ex]{\linewidth}{0.5pt}

%Instructions
\begin{enumerate}
    \item 
        Separate egg yolks from whites before beating eggs. Keep yolks in milk mixture. Fold in stiffly beaten whites into completed batter before cooking on griddle.
\end{enumerate}
\newpage 

% -----------------------------------------------------
\subsection{Hard Boiled Eggs} 
\noindent\rule[0.5ex]{\linewidth}{1pt}

%Ingredients
\begin{framed}
    \begin{itemize}
        \item eggs
        \item water
    \end{itemize}
\end{framed}

%Instructions
\begin{enumerate}
    \item 
        Place eggs in water, add enough water so eggs are coverd by an inch or two of water. The more eggs, the more water (6 eggs - 1 inch water, 7-12 eggs - two inches of water)
    \item 
        Bring water to roiling boil.
    \item 
        Turn off heat, leave pot on hot burner for 10 to 12 minutes (slightly more if lots of water).
    \item 
        Remove eggs from water.
\end{enumerate}
\newpage

% -----------------------------------------------------
\subsection{Lemon and Ginger Scones} 
\noindent\rule[0.5ex]{\linewidth}{1pt}

%Ingredients
\begin{framed}
    \begin{itemize}
        \item 2 cups (260 grams) all purpose flour
        \item 1/3 cup (50 grams) granulated white sugar
        \item 1 teaspoon baking powder
        \item 1/4 teaspoon baking soda
        \item 1/4 teaspoon salt
        \item 1/2 cup (113 grams) (1 stick) unsalted butter, cold and cut into pieces
        \item 1/2 cup (70 grams) crystallized ginger, chopped into small pieces
        \item Zest of 1 large lemon
        \item 2/3 cup (160 ml) buttermilk
    \end{itemize}
\end{framed}

%Instructions
\begin{enumerate}
    \item 
        In a large bowl, whisk together the flour, sugar, baking powder, baking soda and salt. Cut the butter into small pieces and blend into the mixture with a pasty blender or two knives. The mixture should look like coars crumbs.
    \item 
        Stir in the chopped crystalized ginger and lemon zest. Add the buttermilk to flour mixture just until the dough comes together. Do not over mix the dough.
    \item
        Transfer to a lightly floured surface and knead dough gently four or five times and then pat the dough into a circle that is about 7 inches (18cm) round and about 1 1/2 inches (3.75 cm) thick. 
    \item
        Cut the circle in half, then cut each half ito 4 pie shaped wedges (triangles). Place the scones on the baking sheet. Brush the tops of the scones with a little cream.
    \item
        Bake for about 20 to 25 minutes or until golden brown and a toothpick inserted in the middle comes out clean. Transfer to a wire rack to cool (makes 8 scones).
\end{enumerate}
\newpage

% -----------------------------------------------------
\subsection{Ziploc Omelet} 
\noindent\rule[0.5ex]{\linewidth}{1pt}

%Ingredients
\begin{framed}
    \begin{itemize}
        \item 2 eggs per person 
        \item Ingredients to taste (cheese, ham, bacon bits, onion, hash browns, salsa, tomato, peppers, etc.)
    \end{itemize}
\end{framed}

%Instructions
\begin{enumerate}
    \item
        Write each persons name on a quart-sized Ziploc bag. Crack eggs into bag (2 per bag)
    \item
        Shake and combine eggs. Add in optional ingredients. Shake the bag ot combine and zip it up.
    \item 
        Submerse the bags into a large pot of rolling, boiling water for EXACTLY 13 MINUTES. A large pot of water
    \item
        Can usually handle 6-8 omelets. Unzip the bags and roll onto plates
\end{enumerate}
\newpage

% =====================================================
\vspace*{\fill}
\begin{center}
    \section{Soups}
\end{center}
\vspace*{\fill}
\newpage

% -----------------------------------------------------
\subsection{Caldo Verde (Spicy Sausage and Kale Sopa)} 
\noindent\rule[0.5ex]{\linewidth}{1pt}

%Ingredients
\begin{framed}
    \begin{itemize}
        \item 1/4 cup olive oil
        \item 1 cup chopped onion
        \item 2 tsp chopped garlic
        \item 5 cups Ihaho potatoes, peeled and thinly sliced
        \item 1 quart water
        \item 1 quart chicken broth
        \item 6 oz chorizo sausage, thinly sliced
        \item salt and black pepper
        \item 1 lb kale, washed, trimmed of the thick stems and thinly sliced
    \end{itemize}
\end{framed}

%Instructions
\begin{enumerate}
    \item 
        In a medium soup pot, heat 3 tablespoons of olive oil, add onions and garlic and cook for 2 to 3 minutes until they turn glassy, don't let them get brown. 
    \item 
        Add potatoes and water. Cover and boil gently over medium heat for 20 minutes. Meanwhile, in a skillet cook sausage until most of the fat is rendered out. Drain and set aside. 
    \item 
        When the potatoes are tender, mash them with a potato masher right in the pot. Add sausage to the soup then add kale. Simmer for 5 minutes. Add the remaining olive oil and season. Ladle into bowls and serve.
\end{enumerate}
\newpage

% -----------------------------------------------------
\subsection{Chili con Carne} 

%Ingredients
\begin{framed}
    \begin{itemize}
        \item 1 lb ground beef (or turkey)
        \item 1 C chopped onion
        \item 3/4 C green pepper
        \item 1 clove garlic, minced
        \item 1 28 oz can tomatoes, cut up
        \item 2 16 oz cans dark red kidney beans, drained
        \item 1 16 oz can tomato sauce
        \item 3 oz (1/2 can) tomato paste
        \item 1 T chili powder
        \item 1 tsp dried basil, crushed
        \item 1/2 tsp salt
        \item 1/4 tsp pepper
    \end{itemize}
\end{framed}

%Instructions
\begin{enumerate}
    \item
        In a large kettle cook ground beef, onion, green pepper, and garlic until meat is browned. Drain off fat.
    \item
        Stir in undrained tomatoes, kidney beans, tomato sauce, chili powder, basil, salt, and pepper. Bring to boiling; reduce heat.
    \item
        Cover and simmer about 20 minutes
    \item
        Makes 4-6 servings
\end{enumerate}
\newpage

% -----------------------------------------------------
\subsection{Cream of Tomato Soup} 
\noindent\rule[0.5ex]{\linewidth}{1pt}

%Ingredients
\begin{framed}
    \begin{itemize}
        \item 8 medium tomatoes, peeled and seeded
        \item 1 teaspoon basil, fresh, chopped
        \item 1 1/2 cups chicken broth
        \item 1/2 cup onion, chopped
        \item 3 tablespoons butter
        \item 3 tablespoons flour
        \item 1/4 teaspoon salt
        \item 2 dashes pepper
        \item 1 cup 2\% low-fat milk
    \end{itemize}
\end{framed}

%Instructions
\begin{enumerate}
    \item 
        Combine broth, onion, tomato \& basil in a saucepan. Bring to boil, reduce heat, cover \& simmer 15 min.
    \item 
        Blend with hand blender until smooth. Pour into bowl.
    \item
        Melt butter in the saucepan. Stir in the flour, salt \& pepper.
    \item 
        Add milk all at once. Cook \& stir until mixture is thickened and bubbly.
    \item 
        Stir in tomato mixture. Cook \& stir till soup is heated through. Season to taste with more salt \& pepper.
\end{enumerate}
\newpage

% -----------------------------------------------------
\subsection{Creamy Spinach Artichoke Soup} 
\noindent\rule[0.5ex]{\linewidth}{1pt} 

\paragraph Servings: 6; Authors: Melissa Stadler, Modern Honey \& Allison Dennis

%Ingredients
\begin{framed}
\begin{itemize}
    \item 2 tablespoons butter
    \item 1 onion chopped
    \item 4 garlic cloves minced (or 3/4 teaspoon garlic powder
    \item 1 9-ounce package of fresh chopped spinach (may use 3/4 of package, if so desire)
    \item 1 teaspoon salt
    \item 1 teaspoon pepper
    \item 3 tablespoons flour (may delete if you want gluten-free or keto-friendly soup)
    \item 6 cups chicken broth
    \item 1 14-ounce can artichoke hearts drained and roughly chopped
    \item 2 cups heavy cream (depending on how creamy you want the soup)
    \item 1 8 ounce package cream cheese (cubed)
    \item 1 cup parmesan cheese plus additional 1/2 cup for garnish
    \item 1.5 pound chicken, boiled then diced
\end{itemize}
\end{framed}

%Instructions
\begin{enumerate}
\item 
In a pot, boil water
\item 
In a separate large pot, melt butter over medium-high heat. Add onion and saute for 5 minutes. 
\item 
Boil chicken in water pot.
\item 
Add garlic and saute for 1 minute longer.
\item 
Stir in fresh spinach. Stir often and break apart using a wooden sppon. Cook for 5-7 minutes or until spinach is wilted. Sprinkle with salt and pepper. 
\item 
Stir in flour, stirring well to avoid any lumps.
\item 
Pour in chicken broth and artichoke hearts.
\item 
Remove chicken from heat, drain water, and add chicken to soup mixture. Heat for 5-10 minutes.
\item 
Turn heat to LOW. Add heavy cream and stir together (low heat prevents curdling).
\item 
Add cream cheese and let melt slowly while stirring. Takes about 15-20 minutes
\item 
Stir in parmesan cheese.
\item 
Season according to taste. Sprinkle with parmesan cheese shavings
\end{enumerate}
\newpage

% -----------------------------------------------------
\subsection{Creamy Sweet Potato Soup} 
\noindent\rule[0.5ex]{\linewidth}{1pt}

%Ingredients
\begin{framed}
    \begin{itemize}
        \item 2 Tbsp (1/4 stick) butter
        \item 1 cup chopped onion
        \item 2 small celery stalks, chopped
        \item 1 medium leek, sliced (white and pale green parts only)
        \item 1 large garlic clove, chopped
        \item 1 1/2 pounds red-skinned sweet potatoes (yams), peeled, cut into 1 inch pieces (about 5 cups)
        \item 4 cups chicken stock or canned low-salt chicken broth (use vegetable broth for vegetarian option)
        \item 1 cinnamon stick
        \item 1/4 teaspoon ground nutmeg
        \item 1 1/2 cups half and half
        \item 2 Tbspsp maple syrup
        \item The leafy tops of the clery stalks, chopped
    \end{itemize}
\end{framed}

%Instructions
\begin{enumerate}
    \item 
        Melt the butter in a large, heavy-bottomed pot over medium-high heat. Add the chopped onnion and saute for about 5 minutes. Add chopped celery stalks and leek, saute about 5 minutes. Add garlic and saute 2 minutes.
    \item
        Add sweet potatoes, chicken stock, cinnamon stick, and nutmeg; bring to boil. Reduce heat and simmer uncovered until potatoes are tender, about 2o minutes.
    \item
        Remove cinnamon stick and discard. Working in batches, puree soup in blender until smooth. Return to pot.
    \item 
        Add half and half and maple syrup and stir over medium-high heat to heat through. Season soup to taste with salt and pepper. (Can be prepared 1 day ahead. Cool soup slightly. Cover and refrigerate soup and celery leaves separately. Bring soup to simmer before continuing.) Ladle into bowls. Sprinkle with celery leaves.
\end{enumerate}
\newpage

% -----------------------------------------------------
\subsection{Fruit Soup [Christmas]} 
\noindent\rule[0.5ex]{\linewidth}{1pt}

%Ingredients
\begin{framed}
    \begin{itemize}
        \item 1 cup raisins 
        \item 1 cup prunes 
        \item 1/2 cup rice or tapioca 
        \item 1 orange 
        \item 1 lemon 
        \item 1 cup sugar 
        \item 2 cinnamon sticks 
        \item 2 apples
        \item Any fruit juice cocktail 
        \item 2 quarts water 
    \end{itemize}
\end{framed}

%Instructions
\begin{enumerate}
    \item 
        Peel and dice apples. Juice orange and lemon.
    \item 
        Place all ingredients in a large pot. Simmer slowly for several hours stirring occasionally until thickened.
\end{enumerate}
\newpage

% =====================================================
\vspace*{\fill}
\begin{center}
    \section{Breads}
\end{center}
\vspace*{\fill}
\newpage

% -----------------------------------------------------
\subsection{Buttermilk Biscuits} 
\noindent\rule[0.5ex]{\linewidth}{1pt}

%Ingredients
\begin{framed}
    \begin{itemize}
        \item 2 cups flour
        \item 1/2 tsp salt
        \item 5 tbsp shortening
        \item 3 tsp baking powder
        \item 1/4 tsp baking soda
        \item 1 cup buttermilk
    \end{itemize}
\end{framed}

%Instructions
\begin{enumerate}
    \item
        Combine dry ingredients, cut in shortening, add buttermilk.  Knead a couple of times in your hands.
    \item
        Roll out on floured board, cut with biscuit cutter. Bake at 450 for 10-12 minutes.
\end{enumerate}
\newpage

% -----------------------------------------------------
\subsection{Mile High Biscuits} 
\noindent\rule[0.5ex]{\linewidth}{1pt}

%Ingredients
\begin{framed}
    \begin{itemize}
        \item 3 cups flour
        \item 1 tbsp plus 1 tsp baking powder
        \item 1/2 tsp cream of tartar
        \item 1 egg, beaten 
        \item 1/4 cup sugar
        \item 3/4 tsp salt
        \item 1/2 cup shortening
        \item 1 cup plus 2 tbsp milk
    \end{itemize}
\end{framed}

%Instructions
\begin{enumerate}
    \item
        Combine dry ingredients. Cut in shortening. Add egg and milk all at once. Knead 10 0r 12 times.
    \item
        Roll out.k Cut with something smaller than a standard biscuit – like a small juice glass.. Place on ungreased
    \item
        Cookie sheet and freeze. After frozen they can be removed from cookie sheet and put in a Ziploc bag.
    \item
        Bake on lightly greased sheet at 475 for 12-15 minutes.
\end{enumerate}
\newpage

% -----------------------------------------------------
\subsection{Pioneer Cornbread} 
\noindent\rule[0.5ex]{\linewidth}{1pt}

%Ingredients
\begin{framed}
    \begin{itemize}
        \item 1 1/3 C Cold Margerine or Butter (frozen is best)
        \item 1 C sugar
        \item 2 Eggs
        \item 1 C cornmeal
        \item 1 tsp salt
        \item 2 C flour
        \item 1 T baking powder
        \item 2 C milk
    \end{itemize}
\end{framed}


%Instructions
\begin{enumerate}
    \item 
        Cut margerine/butter into small cubes.
    \item 
        Mix margerine/butter with sugar and eggs. Mix well and make sure the cubes are well separated and not clumped together
    \item 
        Add remaining ingredients and stir until moistened
    \item 
        Pour into a greased 9x13 pan or two greased cast iron skillets
    \item 
        Bake at 350 degrees for 20-30 min. Cornbread is done when a toothpick inserted in the middle comes out clean
    \item 
        If desired: remove pan from the oven and turn on the broiler, move the rack to the top position and brown the cornbread under the boiler for 30 seconds or so.  Watch carefully the entire time.  The bread can turn from brown to burned in seconds!
\end{enumerate}
\newpage

% -----------------------------------------------------
\subsection{Southern Pecan Biscuits} 
\noindent\rule[0.5ex]{\linewidth}{1pt}

%Ingredients
\begin{framed}
    \begin{itemize}
        \item 3/4 cup mashed sweet potatoes or yams
        \item 1/2 cup melted margarine
        \item 2 tbsp brown sugar
        \item 1/2 cup milk
        \item 2 cups flour
        \item 3 tsp baking powder
        \item 1 tsp salt
        \item 1/2 cup chopped pecans
    \end{itemize}
\end{framed}

%Instructions
\begin{enumerate}
    \item
        Combine potatoes, margarine, sugar and milk until smooth. Add dry ingredients just until moistemed.
    \item
        Stir in pecans.Roll dough to ½ inch thickness and cut with floured cutter. Bake at 400 for 12-18 minutes.
\end{enumerate}
\newpage

% =====================================================
\vspace*{\fill}
\begin{center}
    \section{Appetizers \& Side Dishes}
\end{center}
\vspace*{\fill}
\newpage

% -----------------------------------------------------
\subsection{Apple Salad Dressig (for waldorf salad)} 
\noindent\rule[0.5ex]{\linewidth}{1pt}

%Ingredients
\begin{framed}
    \begin{itemize}
        \item 2 eggs
        \item 1/4 tsp celery seed
        \item 3 tbsp vinegar
        \item 1/2 cup sugar
        \item 1/2 tsp flour
        \item pinch of salt
        \item 1/4 tsp dry mustard
    \end{itemize}
\end{framed}

%Instructions
\begin{enumerate}
    \item
        Stir together,. Cook until thick. Add cool whip till it’s just how you like it.
\end{enumerate}
\newpage

% -----------------------------------------------------
\subsection{Cranberry Cherry Jello Salad} 
\noindent\rule[0.5ex]{\linewidth}{1pt}

%Ingredients
\begin{framed}
    \begin{itemize}
        \item 2 cups fresh cranberrys
        \item 1 cup sugar
        \item 1 1/2 cup water
        \item 3 oz cherry jello 
        \item 3/4 cup chopped celery 
        \item 1/2 cup chopped nuts 
        \item 1/2 cup chopped apples
    \end{itemize}
\end{framed}

%Instructions
\begin{enumerate}
    \item 
        Boil berries, sugar, and water until berries pop, remove from heat.
    \item 
        Add jello and let cool slightly.
    \item 
        Mix in celery, nuts and apples. Put in fridge to chill for a couple hours.
\end{enumerate}
\newpage

% -----------------------------------------------------
\subsection{Easy Guacamole} 
\noindent\rule[0.5ex]{\linewidth}{1pt}

%Ingredients
\begin{framed}
    \begin{itemize}
        \item 2 avocados
        \item 1 small onion,finely chopped
        \item 1 clove garlic, minced
        \item 1 ripe tomato, chopped
        \item 1 lime, juiced
        \item salt and pepper to taste
    \end{itemize}
\end{framed}

%Instructions
\begin{enumerate}
    \item 
        Peel and mash avocados in a medium serving bowl. Stir in onion, garlic, tomato, lime juice, salt and pepper. Season with remaining lime juice and salt and pepper to taste. Chill for half an hour to blend flavors.
\end{enumerate}
\newpage

% -----------------------------------------------------
\subsection{Five Cup Salad} 
\noindent\rule[0.5ex]{\linewidth}{1pt}

%Ingredients
\begin{framed}
    \begin{itemize}
        \item 11 oz can mandarin oranges
        \item 1 cup coconut
        \item 1 cup sour cream
        \item 8 1/4 can pineapple tidbits
        \item 1 cup marshmallows
    \end{itemize}
\end{framed}

%Instructions
\begin{enumerate}
    \item 
        Lightly mix all ingredients together.
    \item
\end{enumerate}
\newpage

% -----------------------------------------------------
\subsection{Hummus} 
\noindent\rule[0.5ex]{\linewidth}{1pt}

%Ingredients
\begin{framed}
    \begin{itemize}
        \item 15 ounce can (425 grams) chickpeas, also called garbanzo beans
        \item 1/4 cup (59 ml) fresh lemon juice, about 1 large lemon
        \item 1/4 cup (59 ml) well-stirred tahini
        \item Half of alarge garlic clove, minced
        \item 2 tablespoons olive oil, plus more for serving
        \item 1/2 to 1 teaspoon kosher salt, depending on taste
        \item 1/2 teaspoon ground cumin
        \item 2 to 3 tablespoons water
        \item Dash of ground paprika for serving
    \end{itemize}
\end{framed}

%Instructions
\begin{enumerate}
    \item 
        In the bowl of a food processor, combine tahini an dlemon juice. Process for 1 minute Scrape sides and bottom of bowl then turn on and process for 30 seconds. This extra time helps "whip" or "cream" the tahini, making smooth and creamy hummus possible.
    \item 
        Add te olive oil, minced garlic, cumin and the salt to the whipped tahini and lemon juice mixture. Process for 30 seconds, scrape sides and bottom of bowl then process another 30 seconds.
    \item 
        Open can of chicpeaks, drain liquid then rinse well with water. Add half of the chickpeas to the food processor then process for 1 minute. Scrape sides and bottom of bowl, add remaining chickpeas and process for 1 to 2 minutes or until thick and quite smooth.
    \item 
        Most likely the hummus will be too thick or still have tiny bits of chickpea. To fix this, with the food processor turned on, slowly add 2 to 3 tablespoons of water until the consistency is perfect.
    \item 
        Scrape the hummus into a bowl, then drizzle about 1 tablespoon of olive oil over the top and sprinkle the paprika.
\end{enumerate}

\paragraph 
(Note: Store homemade hummus in an airtight container and refigerate up to one week.)
\newpage

% -----------------------------------------------------
\subsection{Indian Style Basmati Rice}
\noindent\rule[0.5ex]{\linewidth}{1pt}

%Ingredients
\begin{framed}
    \begin{itemize}
        \item 1 1/2 cups basmati rice
        \item 2 tablespoons vegetable oil
        \item 1 (2 inch) piece cinnamon stick
        \item 2 pods green cardamom
        \item 2 whole cloves
        \item 1 tablespoon cumin seed
        \item 1 teaspoon salt, or to taste
        \item 2 1/2 cups water
        \item 1 small onion, thinly sliced
    \end{itemize}
\end{framed}

%Instructions
\begin{enumerate}
    \item 
        Place rice into a bowl with enough water to cover. Set aside to soak for 20 minutes.
    \item 
        Heat the oil in a large pot or sacuepan over medium heat. Add the cinnamon stick, cardamom pods, cloves, and cumin seed. Cook and stir for about a minute, then add the onion to the pot. Saute the onion until a rich golden brown, about 10 minutes.
    \item 
        Drain the water from the rice, and stir the pot. Cook and stir the rice for a few minutes, until lightly toasted. Add salt and water to the pot, and bring to boil. Cover, and reduce heat to low. Simmer for about 15 minutes, or until all of th water has been absorbed. Let stand for 5 minutes, then fluff with a fork before serving.
\end{enumerate}
\newpage

% -----------------------------------------------------
\subsection{Red Hot Jello} 
\noindent\rule[0.5ex]{\linewidth}{1pt}

%Ingredients
\begin{framed}
    \begin{itemize}
        \item 1/2 cup water
        \item 1 pkg red jello 
        \item 1/4 cup red hot candies (generally cinnamon candies)
        \item 1 applesauce
    \end{itemize}
\end{framed}

%Instructions
\begin{enumerate}
    \item 
        Boile water and add water to red jello and candies. Dissolve candies.
    \item 
        Mix in applesauce. Chill in fridge for a couple hours.
\end{enumerate}
\newpage

% =====================================================
\vspace*{\fill}
\begin{center}
    \section{Vegetables}
\end{center}
\vspace*{\fill}
\newpage 

% -----------------------------------------------------
\subsection{Coleslaw} 
\noindent\rule[0.5ex]{\linewidth}{1pt}

%Ingredients
\begin{framed}
    \begin{itemize}
        \item 1 head of cabbage, chopped
        \item 1 cup sugar
        \item 1/2 cup apple cider vinegar
        \item 1/4 cup water
        \item 1 tsp celery salt
    \end{itemize}
\end{framed}

%Instructions
\begin{enumerate}
    \item 
        Bring sugar, vinegar, water, and salt to a boil, then boil for 3 minutes to create cabbage dressing.
    \item 
        Mix dressing with chopped cabbage.
\end{enumerate}
\newpage

% -----------------------------------------------------
\subsection{Heather's Brussel Sprouts} 
\noindent\rule[0.5ex]{\linewidth}{1pt}

%Ingredients
\begin{framed}
    \begin{itemize}
        \item 1 lb brussel sprouts
        \item olive oil
        \item minced garlic, 1-2 cloves
        \item salt and pepper to taste
    \end{itemize}
\end{framed}

%Instructions
\begin{enumerate}
    \item
        Preheat oven to 400 degrees
    \item
        Wash brussel sprouts.  Slice in half lengthwise.
    \item
        Toss brussel sprouts with olive oil (1-2 Tbls) , minced garlic, and salt \& pepper in ziploc bag.
    \item
        Spread out sprouts in a single layer on a cookie sheet and bake for 30-45 min,  shaking pan every 5-7 min for even browning.  They will be dark brown to black when done.  Season with additional salt to taste.
\end{enumerate}
\newpage

% =====================================================
\vspace*{\fill}
\begin{center}
    \section{Entrees}
\end{center}
\vspace*{\fill}
\newpage

% -----------------------------------------------------
\subsection{Chef John's Chicken Tikka Masala} 
\noindent\rule[0.5ex]{\linewidth}{1pt}

%Ingredients
\begin{framed}
    \begin{itemize}
        \item 1 1/2 pounds skinless, boneless chicken thighs
        \item 1 tablespoon vegetable oil
        \item 2 teaspoons kosher salt
        \item 2 teaspoons garam masala
        \item 2 teaspoons ground cumin
        \item 1 teaspoon ground coriander
        \item 1 teaspoon smoked paprika
        \item 1 teaspoon ground turmeric
        \item 1/2 teaspoon ground black pepper
        \item 1/4 teaspoon cayenne pepper
        \item 1/8 teaspoon ground cardamom
        \item 2 tablespoons clarified butter (ghee), or more as needed
        \item 1 onion, chopped
        \item 1/4 cup tomato paste
        \item 4 cloves garlic, finely grated
        \item 1 tablespoon finely grated ginger, or more to taste
        \item 1 cup crushed tomatoes
        \item 1 (13 ounce) can coconut milk
        \item 1/2 cup chicken broth, or as needed
        \item 2 tablespoons chopped fresh cilantro
        \item 1/2 teaspoon red pepper flakes
        \item salt and ground black pepper to taste
    \end{itemize}
\end{framed}

\paragraph
(Note: It is recommended to eat this entre with a side dish of "Indian Style Basmati Rice" (see Appetizers \& Side Dishes for recipe).)

%Instructions
\begin{enumerate}
    \item 
        Place chicken in a bowl and drizzle vegetable oil over chicken; toss to coat.
    \item
        Whisk kosher salt, garam masala, ground cumin, ground coriander, smoked paprika, ground turmeric, black pepper, cayenne pepper, and cardamom together in a small bowl. Season chicken with spice mixture and turn to coat evenly.
    \item
        Melt clarified butter in a large, heavy skillet over high heat. Cook chicken thigs in hot butter until browned on all sides, 5 to 10 minutes. Transfer chicken to a plate. When coo enough to handle, cut chicken into bite size pieces.
    \item
        Reduce heat under the skillet to medium-high. Stir onion into skillet; saute until onion softens and turns translucent, 5 to 6 minutes. Add tomato paste and stir. Saute until paste caramelizes, about 5 minutes. Stir garlic and ginger into tomato-onion mixture and cook until fragrant, about 1 minute.
    \item
        Pour the crushed tomatoes into the skillet and bring to a boil while scraping the browned bits of food off the bottom of the skillet with a wooden spoon. Pour in coconut milk and chicken broth; bring to a simmer, reduce heat to medium low, and cook, stirring occasionally, until flavors blend and sauce is slightly reduce, about 15 minutes.
    \item
        Stir chicken, any accumulated juices from the chicken, cilantro, and red pepper flakes into tomato mixture; bring to a simmer, reduce heat to medium-low, and cook until chicken is cooked through and tender, 10 to 15 minutes. Season with salt and black pepper
\end{enumerate}

\subsubsection{Clarified Butter (Ghee)}
\noindent\rule[0.5ex]{\linewidth}{0.5pt}

%Ingredients
\begin{framed}
    \begin{itemize}
        \item Amount of butter desired, cubed (some mass is lost in this process)
    \end{itemize}
\end{framed}

%Instructions
\begin{enumerate}
    \item 
        In a small saucepan, melt butter over medium-high heat
    \item 
        Continue to cook over medium-high heat; an even layer of white milk proteins will float to the surface.
    \item 
        Bring to a boil; the milk proteins will become foamy.
    \item 
        Lower heat to medium and continue to gently boil; the milk proteins will break apart.
    \item 
        As the butter gently boils, the milk proteins will eventually sink to the bottom of the pot, and the boiling will begin to calm and then cease.
    \item 
        Once boiling has stopped, pour butter through a cheescloth-lined strainer or through a coffee filter into a heatproof containerto remove browned milk solids. Let cool, then transfer to a sealed container and refrigerate until ready to use. Clarified butter should keep for at least 6 months in the refrigerator.
\end{enumerate}
\newpage 


% -----------------------------------------------------
\subsection{Gourmet Four Cheese Macaroni and Cheese} 
\noindent\rule[0.5ex]{\linewidth}{1pt}

%Ingredients
\begin{framed}
    \begin{itemize}
        \item 1 lb rotelle pasta (they look like wagon wheels)
        \item 3/4 lb sharp cheddar cheese, shredded
        \item 1/2 lb gruyer cheese, shredded
        \item 1/2 cup asiago cheese, shredded
        \item 1/2 cup Fontina cheese, shredded
        \item 3 tablespoons unsalted butter
        \item 3 tablespoons all-purpose flour
        \item 2 cups milk (2\%)
        \item 1 teaspoon onion powder
        \item 1/2 teaspoon salt
        \item 1/4 teaspoon dried mustard
        \item 1/4 teaspoon nutmeg
        \item 1/4 teaspoon ground cayenne pepper
        \item 1 cup panko breadcumbs (japanese bread crumbs)
    \end{itemize}
\end{framed}

%Note
\paragraph 
(This is a BAKED macaroni and cheese, therefore, it will NOT turn out with a lot of extra cheesy sauce, as it is absorbed into the pasta while baking. If you requre and extra saucy mac \& cheese, just reduce the amount of pasta you place into the dish or make more sauce so it is creamier.)

%Instructions
\begin{enumerate}
    \item 
        Heat oven to 350 degrees. Coat a 3 quart rectangular baking dish with non-stick spray. Bring a large pot of lightly salted water to boiling.
    \item 
        Toss all the shredded cheeses together in a large bowl, set aside.
    \item 
        Melt butter in a medium-sized saucepan over medium heat. Whisk in the flour until smooth and slightly bubbly.
    \item 
        In a thin stream, whisk in the milk. Stir in the onion powder, salt, nutmeg, dried mustard and cayenne.
    \item 
        Bring to a boil over medium high heat. Reduce heat and simmer 3 minutes. Remove from heat; whisk in a 2 1/2 cups of the cheese mixture and stir until smooth. Cover to retain heat.
    \item  
        Once the water boils, add pasta. Cook until your desired doneness, then drain. In the pasta container stir together the cooked pasta and cheese sauce.
    \item 
        Pour half of the mixture into the prepared dish. Sprinkle with a generous cup of the reserved cheeese. Spoon remaining chees-covered pasta into the dish and top with the remaining cheese.
    \item 
        Add 1 cup of japanese panko bread crumbs to the top of the mixture.
    \item 
        Bake at 350 degrees for 30 minutes or until the pank crumbs are lightly browned and the cheese is bubbly. Cool slightly before serving.
\end{enumerate}
\newpage 

% -----------------------------------------------------
\subsection{Karen Martin's Teriyaki Marinate} 
\noindent\rule[0.5ex]{\linewidth}{0.5pt}

%Ingredients
\begin{framed}
    \begin{itemize}
        \item 1/4 cup soy sauce
        \item 1/4 cup vegetable oil
        \item 2 tbsp white wine
        \item 1 tsp sugar
        \item 1/2 tsp ginger
        \item 1 garlic clove, crushed
        \item optional, 1 tsbp cornstarch to thicken
    \end{itemize}
\end{framed}

%Instructions
\begin{enumerate}
    \item 
        Mix all ingredients together. Use as a meat marinate.
\end{enumerate}
\newpage

% -----------------------------------------------------
\subsection{Grandma's Swedish Meatballs [Christmas]} 
\noindent\rule[0.5ex]{\linewidth}{1pt}

%Ingredients
\begin{framed}
    \begin{itemize}
        \item 1/2 lb beef
        \item 1/2 lb veal
        \item 1/2 lb ground pork
        \item 3 tbsp parsley
        \item 1 tsp lemon juice
        \item 1/4 tsp parika
        \item 1/8 tsp allspice
        \item 1/4 tsp salt
        \item 1/2 tsp grated lemon on rind
        \item 1/3 tsp nutmeg
    \end{itemize}
\end{framed}

\paragraph 
(This recipe is primarily for use with Lefse as an alternative for meatballs in that recipe. See "Norwegian Lefse and Meatballs" for instructions.)

%Instructions
\begin{enumerate}
    \item
        Shape meat into 1 ½ inch balls and brown in 2 tbsp butter. Simmer closely coverd until done – about 15 min.
    \item
        In 2 cups stock. Make gravy and season with sherry or 1 or 2 tbsp fresh dil.
\end{enumerate}
\newpage

% -----------------------------------------------------
\subsection{Norwegian Lefse and Meatballs [Christmas]} 
\noindent\rule[0.5ex]{\linewidth}{1pt}

%Ingredients
\begin{framed}
    \paragraph
    (Lefse)
    \begin{itemize}
        \item 4 cups brown, peeled potatoes
        \item 1/2 cup butter (alternate: 1/4 cup butter and 1/4 cup crisco)
        \item 1 1/2 cups flour
        \item 1 teaspoon Salt
    \end{itemize}
    \paragraph
    (Meatballs)
    \begin{itemize}
        \item 2 eggs, beaten
        \item 1 cup milk
        \item 1 cup dry bread crumbs
        \item 1/2 cup minced onion
        \item 2 teaspoons salt
        \item 2 teaspoons sugar
        \item 3/4 teaspoon ground ginger
        \item 3/4 teaspoon ground nutmeg
        \item 3/4 teaspoon ground allspice
        \item 1/4 teaspoon pepper
        \item 2 pounds lean ground beef
        \item 1 pound ground pork
    \end{itemize}
    \paragraph
    (Gravy)
    \begin{itemize}
        \item 3 tablespoons butter
        \item 2 tablespoons minced onioin
        \item 5 tablespoons all-purpose
        \item 4 cups beef broth
        \item 1/2 cup heavy whipping cream
        \item Dash of cayenne pepper
        \item Dash of pepper
    \end{itemize}
\end{framed}

%Instructions
\paragraph
(Lefse)
\begin{enumerate}
    \item 
        Boil potatoes till tender.
    \item 
        Put potatoes through a potato ricer or food mill (makes potato into a very fine paste).
    \item 
        Mix in butter and salt while potatoes are still hot. Slowly add in flour.
    \item 
        The following formation and cooking process should be done by a 2 person team, the first person forms the lefse, the second person cooks and flips the lefse. Otherwise the process will take a very long time. 
    \item 
        One at a time: Gently form into a shaped walnut sized ball (do not over handle). Roll on heavily floured surface (can roll on floured pastry cloth or tea towel). Flatten AS THIN AS POSSIBLE using a heavily floured rolling pin (best tool for this is a Lefse Rolling pin, or, if desparate, a rolling pin with a cloth sleeve. Whichever rolling pin used should be heavily floured and refloured between rolling each lefse).
    \item 
        Fry in a dry frying pan in a medium high heat. Look for little bubbles in the lefse, when the bubbles are evenly distributed it's time to flip it (the cooked side should have little light brown spots). Cooking per side should only take 10 to 20 seconds, if it's taking longer the griddle is probably not hot enough or the lefse is too thick. Once both sides are cooked, set aside on paper towels on a plate. (Manipulation of lefse is easiest with a Lefse stick, essentially a flattened, narrow ended stick).
    \item 
        It's a good idea to seperate every dozen or so lefse by paper towel when stacking, for ease of handling and storage. Store in an airtight container (Can be frozen once cooled. To thaw, simply leave them on the counter till room temperature.).
\end{enumerate}
\paragraph
(Meatballs \& Gravy)
\begin{enumerate}
    \item 
        In a mixing bowl, combine eggs, milk, bread crumbs, onion and seasonings. Let stand until crumbs absorb milk. Add meat; stir until well blended. Shape into 1 inch meatballs. Place on a greased jelly-roll pan. Bake at 400 degrees until browned, about 18 minutes. Set aside.
    \item 
        For gravy, melt butter over medium-high heat ina  large skillet. Saute onion until tender. Stir in flour and brown lightly. Slowly add broth; cook and stir until smooth and thickened. Blend in cream, cayenne pepper and black pepper. Gently stir in meatballs; heat through but do not boil.
\end{enumerate}
\newpage

% -----------------------------------------------------
\subsection{Red, White, and Blue Mac and Cheese} 
\noindent\rule[0.5ex]{\linewidth}{1pt}

%Ingredients
\begin{framed}
    \begin{itemize}
        \item 3 tablespoons butter
        \item 1/4 cup flour
        \item 1 1/2 cups half and half
        \item 1 1/2 cups heavy cream
        \item 2 cloves garlic, finely grated
        \item 1/4 teaspoon cayenne pepper
        \item 1 teaspoon Worcestershire sauce
        \item Salt an dblack pper to taste
        \item 1 cup sun-dried tomatoes, drained and sliced
        \item 1 3/4 cups (7 ounces) Wisconsi white chedder cheese, shredded and divided
        \item 1 3/4 cups (7 ounces) Wisconsin montery jack cheese, shredded and divided
        \item 1 1/2 cups (9 ounces) Wisconsin blue cheese, crumbled and dividied
        \item 1 pound cavatappi pasta cooked to al dente, drained and cooled
        \item paprika, optional
    \end{itemize}
\end{framed}

%Instructions
\begin{enumerate}
    \item 
        Preheat oven to 350 degrees.
    \item 
        In a medium pot over medium heat, melt butter. Add flower and whisk for 3 minutes until lightly browned.
    \item 
        Add heavy whipping cream, half and half, garlic, cayenne, Worcestershire sauce, salt and pepper. Stir for about 3 minutes over medium heat until slightly thick.
    \item 
        Remove from heat and add sun dried tomatoes and cheeses, reserving 1/2 cup whit chedder, 1/2 cup monterey jack, and 1/2 cup blue cheese. Stir until chees is melted into sauce.
    \item 
        Butter 13x9 baking dish. In lare pot, add sauce to pasta. Stir until pasta is evenly coated. Pour into baking dish and spread evenly. Sprinkle reserved cheese over top. Sprinkle with paprika if desired.
    \item 
        Bake 25-30 miutes until bubbly. Le tsit 5-10 minutes before serving.
\end{enumerate}
\newpage

% -----------------------------------------------------
\subsection{Teresa Gay's Sweet and Sour Chicken} 
\noindent\rule[0.5ex]{\linewidth}{1pt}

%Ingredients
\begin{framed}
    \begin{itemize}
        \item 12 boneless, skinless chicken thighs, cut in half lengthwise (can also use wings)
        \item 2 tsp garlic powder
        \item 1 tsp salt
        \item 1/2 tsp pepper
        \item 1 cup conrstarch
        \item 2 eggs
        \item 1 tbsp water
        \item 2 cups chicken broth
        \item 1 cup ketchup
        \item 1 1/2 cups brown sugar
        \item 1 cup apple cider or balsamic vinegar
        \item 2 tbsp soy sauce
    \end{itemize}
\end{framed}

%Instructions
\begin{enumerate}
    \item 
        Mix together garlic powder, salt, pepper and cornstarch. Set aside.
    \item 
        In a separate bowl, beat together eggs and water. Set aside.
    \item  
        For the sauce, combine chicken broth, ketchup, brown sugar, apple cider/balsamic vinegar and soy sauce. Simmer together about 30 minutes.
    \item 
        Dip chicken in egg/water mixture, then in cornstarch and seasoning mixture.
    \item 
        Brown chicken in hot oil an dlay in a shallow pan lined with foil.
    \item 
        Pour the cooked sauce over the chicken. It shouldn't totally cover the chicken.
    \item 
        Bake uncovered 45 min to 1 hour at 325 degrees F, basting wiht sauce several times.
    \item 
        You can brown chicken and freeze in pan. Thaw and continue by adding sauce and baking in the oven. Serve with white rice and sauce in side bowl.
\end{enumerate}
\newpage

% =====================================================
\vspace*{\fill}
\begin{center}
    \section{Deserts}
\end{center}
\vspace*{\fill}
\newpage

% -----------------------------------------------------
\subsection{Banana Cupcakes with Honey Cinnamon Frosting} 
\noindent\rule[0.5ex]{\linewidth}{1pt}

%Ingredients
\begin{framed}
    \begin{itemize}
        \item 1 1/2 cups all-purpose flour, (spooned and leveled)
        \item 3/4 cup sugar
        \item 1 teaspoon baking powder
        \item 1/2 teaspoon baking soda
        \item 1/4 teaspoon salt
        \item 1/2 (1 stick) unsalted butter, mleted
        \item 1 1/2 cups mashe bananas (about 4 ripe bananas), plus 1 whole banana, for garnish (optional)
        \item 2 large eggs
        \item 1/2 teaspoon pure vanilla extract
        \item Honey-Cinnamon Frosting
    \end{itemize}
\end{framed}

%Instructions
\begin{enumerate}
    \item 
        Preheat oven to 350 degrees. Line a standard 12-cup muffin pan with paper liners. In a medium bowl, whisk together flour, sugar, baking powder, baking soda, and salt.
    \item 
        Make a well in center for flour mixture. In well, mix together butter, mashed bananas, eggs, and vanilla. Stir to incorporate flour mixture (do not overmix). Dividing evenly, spoon batter into muffin cups.
    \item 
        Bake until a toothpick inserted in center of a cupcake comes out clean, 25 to 30 minutes. Remove cupcakes from pan: cool completely on a wire rack. Spread tops with Honey-Cinnamon Frosting.
    \item 
    \item 
\end{enumerate}

\subsubsection{Honey Cinnamon Frosting}
\noindent\rule[0.5ex]{\linewidth}{0.5pt}

%Ingredients
\begin{framed}
    \begin{itemize}
        \item 1 1/4 cup confectioner's sugar 
        \item 1/2 cup (1 stick) unsalted butter, room temperature
        \item 1 tablespoon honey
        \item 1/8 teaspoon ground cinnamon
    \end{itemize}
\end{framed}

%Instructions
\begin{enumerate}
    \item 
        In a medium bowl, using an electric mixer, beat confectioners' sugar, unsalted butter, honey, and ground cinnamon until smooth, 4 to 5 minutes
\end{enumerate}
\newpage

% -----------------------------------------------------
\subsection{Banana Pudding} 
\noindent\rule[0.5ex]{\linewidth}{1pt}

%Ingredients
\begin{framed}
    \begin{itemize}
        \item 1 package of instant vanilla pudding
        \item 2ish (just yellow) bananas
        \item 3 servings of vanilla wafers
        \item 1 can of cool whip (whipped cream)
    \end{itemize}
\end{framed}

%Instructions
\begin{enumerate}
    \item 
        Prepare instant pudding according to directions on box
    \item 
        Slice bananas
    \item 
        Place vanilla wafers in a plastic bag and crush until they are crumbs.
    \item 
        Fill the bottom of a cup with pudding, about an inch thick. Sprink some wafer crumbs on top of pudding. Add a layer of banana. Add a thin layer of cool whip. Repeat layers until glass is full. Top off with cool whip.
\end{enumerate}
\newpage

% -----------------------------------------------------
\subsection{Better Than Anything Toffee Recipe} 
\noindent\rule[0.5ex]{\linewidth}{1pt}

%Ingredients
\begin{framed}
    \begin{itemize}
        \item 1 cup coarsley chopped pecans
        \item 1 cup (2 sticks) unsalted butter
        \item 1 cup granulated sugar
        \item 1/2 tsp kosher salt
        \item 1 tsp vanilla extract
        \item 1 cup milk chocolate chips
    \end{itemize}
\end{framed}

%Instructions
\begin{enumerate}
    \item 
        Spray a 9-inch square baking dish with cooking spary and line with parchment paper.
    \item 
        Spread the chopped pecans in a single layer on top of the parchment.
    \item 
        Add butter, sugar, and salt to a heavy bottomed 3 quart pot
    \item 
        Bring to a boil over medium low heat, stirring frequently.
    \item 
        Once the candy is boiling, stir frequently, slowly and evenly, until the candy has reached 290F to 300F, or "hard crack" on a candy thermometer.
    \item
        Once the candy has reached 290F-300F, remove from heat and gently stir in the vanilla extract.
    \item 
        Carefully pour the mixture over the chopped pecans.
    \item 
        Let the candy sit for a few minutes, undisturbed, before sprinkling the chocolate chips over the top.
    \item 
        Cover the baking dish with foil and let sit for 5 minutes or until the chocolate has softened.
    \item 
        Remove the foil and gently spread the softened chocolate into an even layer. An offset spatula works best for this.
    \item
        Place the candy in the refigerator and let cool completely. Give it at least 2 hours.
    \item 
        Lift the parchment out of the baking dish and place the toffee on a cutting board or solid surface.
    \item 
        Use a knife to gently break it into smaller pieces.
    \item 
        Store in an airtight container in a cool place.
\end{enumerate}
\newpage

% -----------------------------------------------------
\subsection{Cake Pops} 
\noindent\rule[0.5ex]{\linewidth}{1pt}

%Ingredients
\begin{framed}
    \begin{itemize}
        \item
            1 box of cake mix, bakers choice (my favorite is cherry chip)
        \item 
            1 container of icing, bakers choice (my favorite is vanilla)
        \item 
            1 box of almond bark melting chocolate (my favorite is white chocolate)
        \item 
            Sprinkles (optional)
        \item 
            Cake Pop Sticks
    \end{itemize}
\end{framed}

%Instructions
\begin{enumerate}
    \item 
        Bake cake according to directions on box (additional ingredients may be needed to bake cake, such as eggs and/or oils).
    \item 
        Let cake cool for about 30 minutes.
    \item 
        Once cake is cooled, using hands or food processor, crumble cake into crumbs.
    \item 
        Mix half of the icing container with the cake crumbs. If the mixture is still crumbly, add more icing. Mixture should be sticky but not overly moist.
    \item 
        Roll tablespoon helping of cake mixture into balls. Place balls on a lined cookie sheet and refrigerate for one hour.
    \item 
        Using a chocolate melting pot, warm half of the chocolate. Remove cake balls from refrigerator and place cake pop stick in each ball.
    \item 
        Carefully dip cake ball into chocolate, spin stick slowly to make sure all side of pop are covered. Gently tap off extra chocolate back into melting pot. If sprinkles are desired, add right after cake pop is remove from the chocolate.
    \item 
        Let all cake pops set for 30 minutes before packaging or moving.
    \item 
        Place stick end of cake pop into a foam board for drying.
\end{enumerate}
\paragraph
(Tips: If chocolate is too thick, it will pull the cake ball apart. To thin chocolate mixture, add one teaspoon of crisco at a time and mix chocolate. Repeat this process until the chocolate is a thinner consistency and transfer to cake ball easily.)
\newpage

% -----------------------------------------------------
\subsection{Chocolate Chip Cookie Cheesecake} 
\noindent\rule[0.5ex]{\linewidth}{1pt}

%Ingredients
\begin{framed}
    \begin{itemize}
        \item 3 (8 ounce) packages cream cheese, softened
        \item 3 eggs
        \item 3/4 cup sugar
        \item 1 teaspoon vanilla extract
        \item 2 (16.5 ounce) rolls refrigerator chocolate chip cookie dough (keep refrigerated until needed)
    \end{itemize}
\end{framed}

%Instructions
\begin{enumerate}
    \item 
        Preheat oven to 350 degrees F.
    \item 
        In a large bowl, beat together cream cheese, eggs, sugar, and vanilla extract until well mixed. Set aside.
    \item 
        Slice cookie dough rolls in 1/4 inch slices. Arrange slices from one rool on bottom of a greased 9x13 inch glass baking dish; press together so there are no holes in dough. Spoon cream cheese mixture evenly over dough; top with remaining slices of cookie dough.
    \item 
        Bake 45 to 50 minutes, or until golden and center is slightly firm.
    \item 
        Remove from oven, let cool, then refrigerate. Cut into slices when well chilled.
    \item 
        If desired top with ice cream or whipped cream.
\end{enumerate}
\newpage

% -----------------------------------------------------
\subsection{Cranberry Orange Shortbread Cookies} 
\noindent\rule[0.5ex]{\linewidth}{1pt}

%Ingredients
\begin{framed}
    \begin{itemize}
        \item 1/2 cup dried carnberries (Craisins)
        \item 3/4 cups sugar, divided
        \item 2 1/2 cups all purpouse flour - spooned and leveled, not scooped
        \item 1 cup butter, cubed (and cold)
        \item 1 tsp almond extrat
        \item zest of 1 orange
        \item 1 to 2 tbsp of fresh orange juice (optional)
        \item additoinal sugar to coat cookies before baking if desired
    \end{itemize}
\end{framed}

%Instructions
\begin{enumerate}
    \item  
        Line a baking sheet with parchment paper and set aside.
    \item 
        Combine cranberries and 1/4 cup of sugar in a food processor and proces sjust until the cranberries are broken down into smaller pieces. Set aside.
    \item  
        Combine flour and remaining sugar in a large bowl.
    \item 
        Use a pastry cutter to cut in butter. You want very fine crumbs.
    \item 
        Stir in extract, cranberries and sugar mixture, orange zest and orange juice (optional)
    \item  
        Use your hands to knead the dough until it comes together and forms a ball. Work the dough until comes together.
    \item  
        Shape the dough into a log about two inches in a diameter and wrap in plastic wrap. Refrigerate for two hours or up to 72 hours.
    \item 
        Preheat oven to 325F.
    \item  
        Cut slices of cookie dough about 1/4 inch thick
    \item  
        Place about a half a cup of sugar in a bowl and coat the cookie slices with sugar.
    \item  
        Place cookies on baking sheet and bake for 12 to 15 minutes or just until cookies are set. *Do not over bake* (I pull mine at 12 minutes).
    \item 
        Let cookies cool for several minutes on baking sheet before removing to cooling rack. Let cool completely.
    \item  
        Store in airtight container for 3 days or freeze for up to 3 months.
\end{enumerate}
\newpage

% -----------------------------------------------------
\subsection{Crock Pot Candy} 
\noindent\rule[0.5ex]{\linewidth}{1pt}

%Ingredients
\begin{framed}
    \begin{itemize}
        \item 32 oz. mixed nuts with extra cashews 
        \item 12 oz. semi sweet chocolate chips 
        \item 1 bakers 4 ozs. german chocolate bar
        \item 32 oz white chocolate chips
    \end{itemize}
\end{framed}

%Instructions
\begin{enumerate}
    \item 
        Put mixed nuts in crockpot. 
    \item 
        Pour chocolate chips over nuts.
    \item 
        Break german chocolate bar over top.
    \item 
        Pour white chocolate chips over everything.
    \item 
        Cook on low (DO NOT REMOVE LID) for 2 hours.
    \item 
        Stir well and drop on waxed paper to cool.
\end{enumerate}
\newpage

% -----------------------------------------------------
\subsection{Dark Molasses Cookies} 
\noindent\rule[0.5ex]{\linewidth}{1pt}

%Ingredients
\begin{framed}
    \begin{itemize}
        \item 3/4 cup shortening
        \item 1 cup white sugar
        \item 1 egg
        \item 1/4 cup molasses
        \item 2 cup sifted flour
        \item 2 tsp baking soda
        \item 1 tsp cinnamon
        \item 3/4 tsp ginger
        \item 1/2 tsp ground cloves
        \item 1/4 tsp salt
    \end{itemize}
\end{framed}

%Instructions
\begin{enumerate}
    \item 
        Combine shortening, sugar, egg, and molasses. Mix thoroughly.
    \item 
        Mix in flour, baking soda, cinnamon, ginger, ground cloves, and salt.  
    \item 
        Make batter in small balls of dough using your hands. Roll balls in cinnamon and sugar mixture.
    \item 
        On greased baking sheet, bake 10 or 12 minutes at 350 degrees.
\end{enumerate}
\newpage

% -----------------------------------------------------
\subsection{Dutch Apple Pie} 
\noindent\rule[0.5ex]{\linewidth}{1pt}

%Ingredients
\begin{framed}
    \begin{itemize}
        \item Your favorite homemade pie crusts or 1 ready-made pie crust
        \item 5 1/2 cups peeled cored sliced cooking apples
        \item 1 tablespoon lemon juice
        \item 1/4 cup brown sugar, packed
        \item 3 tablespoons flour
        \item 1/2 teaspoon ground cinnamon
        \item 1/4 teaspoon nutmeg
    \end{itemize}
    \paragraph
    (Topping)
    \begin{itemize}
        \item 3/4 cup flour
        \item 1/4 cup granulated sugar
        \item 1/4 cup brown sugar, packed
        \item 1/3 cup butter or 1/3 cup margarine, room temperature
    \end{itemize}
\end{framed}

%Instructions
\begin{enumerate}
    \item 
        Preheat oven to 375 degree F.
    \item
        Fit pie crust into pie plate.
    \item 
        In a large bowl, mix sliced apples, lemon juice, both sugars, flour, cinnamon and nutmeg.
    \item 
        Pile into crust.
    \item
        Prepare topping: In a medium bowl, with a pastry blender or a fork, mix flour, both sugars, and butter until coarsley crumbled.
    \item
        Sprinkle evenly over apples.
    \item 
        Bake at 375 degrees F for 50 minutes
\end{enumerate}
\newpage

% -----------------------------------------------------
\subsection{English Toffee Sandie Cookies [Christmas]} 
\noindent\rule[0.5ex]{\linewidth}{1pt}

%Ingredients
\begin{framed}
    \begin{itemize}
        \item 1 cup butter, softened (no substitutes)
        \item 1 cup sugar
        \item 1 cup powedered sugar
        \item 1 cup vegetable oil
        \item 2 eggs 
        \item 1 tsp almond extract
        \item 3 1/2 cups flour
        \item 1 cup whole wheat flour
        \item 1 tsp cream of tartar
        \item 1 tsp baking soda
        \item 1 tsp salt
        \item 1 8-oz package chocolate covered English toffee bits (Heath) - I use the whole 12 oz. pkg
        \item Additional sugar
    \end{itemize}
\end{framed}

%Instructions
\begin{enumerate}
    \item 
        Cream butter and sugars. 
    \item 
        Add oil eggs \& almond extract. Mix well.
    \item 
        Combine flours, baking soda, cream of tartar and salt. Gradually add to creamed mixture. 
    \item 
        Stir in the toffee bits.
    \item 
        Shape into 1 inch balls (bigger is not better with these). 
    \item 
        Roll in sugar and place on cookie sheet. Flatten with a fork.
    \item 
        Bake 350 degrees F for 9 to 10 minutes or until lightly browned (*really* lightly browned)
\end{enumerate}
\newpage

% -----------------------------------------------------
\subsection{Giant Ginger Cookies} 
\noindent\rule[0.5ex]{\linewidth}{1pt}

%Ingredients
\begin{framed}
    \begin{itemize}
        \item 4 1/2 cups flour
        \item 1 1/2 cups crisco (don't use margerine or butter)
        \item 2 cups sugar
        \item 2 eggs
        \item 1/2 cup molasses
        \item 3/4 cup coarse sugar crystals
        \item 4 tsp ground ginger
        \item 2 tsp baking soda 
        \item 1 1/2 tsp cinnamon
        \item 1 tsp ground cloves
        \item 1/4 salt
    \end{itemize}
\end{framed}

%Instructions
\begin{enumerate}
    \item
        Combine flour, ginger, soda, cinnamon, cloves, salt. Set aside.
    \item
        Beat shortening till soft. Add sugar and beat until fluffy. Add eggs and molasses. Beat well.
    \item
        Add half of flour mixture, then 2nd half.
    \item
        Shape dough in walnut size (or a little bigger) balls. Roll in coarse sugar. Bake on ungreased cookie sheet
    \item
        At 350 for 12-14 minutes till brown and puffed. DO NOT OVERBAKE or they won’t be chewy.
    \item
    \item
    \item
\end{enumerate}

\subsubsection{Chocolate Sauce} 
\noindent\rule[0.5ex]{\linewidth}{0.5pt}
%Instructions
\begin{enumerate}
    \item
        In saucepan melt ¾ cup sugar, ¼ cup butter, 2 oz. unsweetened chocolate squares, 2 tbsp corn syrup, dash salt.
    \item
        When melted add ¼ cup milk, Bring to a boil stirring constantly. Remove from heat. Add 2 tsp,. vanilla.
\end{enumerate}
\newpage

% -----------------------------------------------------
\subsection{Magnolia's Chocolate Cupcakes} 
\noindent\rule[0.5ex]{\linewidth}{1pt}

%Ingredients
\begin{framed}
    \begin{itemize}
        \item 2 cups all purpose flower
        \item 1 teaspoon baking soda
        \item 1 cup (2 sticks) unsalted butter, softened
        \item 1 cup granulated suger
        \item 1 cup firmly packed light brown suger
        \item 4 large eggs, at room temperature
        \item 6 ounces unsweetened chocolate, melted (see note)
        \item 1 cup buttermilk
        \item 1 teaspoons vanilla extract
        \item Vanilla Buttercream (recipe follows) or Chocolate Buttercream
    \end{itemize}
\end{framed}

%Instructions
\begin{enumerate}
    \item  
        Preheat oven to 350 degrees.
    \item 
        Line two 12-cup muffin tins with cupcake papers. Set aside
    \item  
        In a small bowl, sift together the flour and baking soda. Set asid.
    \item  
        In a large bowl, on the medium speed of an electric mixer, cream the butter until smooth. Add the sugars and beat until fluffy, about 3 minutes. Add the eggs, one at a time, beating well after each addition. Add the chocolate, mixing until well incorporated
    \item 
        Add the dry ingredients in three parts, alternating with the buttermilk and vanilla. With each addition, beat until the ingredients are incorportaed, but do not overbeat. 
    \item 
        Using a rubber spatula scrape down the batter in the bowl to make sure the ingredients are well blended and the batter is smooth. Caerfully spoon the batter into the cupcake liners, filling them about three-quarters full. 
    \item 
        Bake for 20-25 minutes, or until a caker tester inserted inthe center of the cupcake comes out clean.
    \item 
        Cool the cupcakes in th etins for 15 minutes. Remove from the tins and cool completely on a wire rack before icing. 
    \item 
        Ice the cupcakes either with Vanilla Buttercream or Chocolate Buttercream
\end{enumerate}

\paragraph 
(Note: If you would like to make a layer cake instead of cupcakes divide the batter between two 9-inch round cake pans and bake the layers for 30-40 minutes)

\subsubsection{Magnolia's Vanilla Buttercream Frosting}
\noindent\rule[0.5ex]{\linewidth}{0.5pt}

%Ingredients
\begin{framed}
    \begin{itemize}
        \item 1 cup (2 sticks) unsalted butter, softed
        \item 6 to 8 cups confectioners' sugar
        \item 1/2 cup milk
        \item 2 teaspoons vanilla extract
    \end{itemize}
\end{framed}

%Instructions
\begin{enumerate}
    \item 
        Place the butter in a large mixing bowl. Add 4 cups of the sugar and then the milk and vanilla. On the medium speed of an electric mixer, beat until smooth and creamy, about 3-5 minutes.
    \item 
        Gradually add the remaining sugar, 1 cup at a time, beating well after each addition (about 2 minutes), until the icing is thick enough to be of good spreading consistency. You may not need to add all of the sugar.
    \item 
        If desired, add a few drops of food coloring and mix thoroughly. (Use and store the icing at room temperature because icing will set if chilled.) Icing can be stored in an airtight container for up to 3 days
\end{enumerate}

\paragraph
(Note: if you are icing a 3-layer cake, use the folloing recipe proportions:
\begin{itemize}
    \item 1 1/2 cups (3 sticks) unsalted butter
    \item 8 to 10 cups confectioners' sugar
    \item 3/4 cup milk
    \item 1 tablespoon vanilla extract
\end{itemize}
)

\subsubsection{Magnolia's Chocolate Buttercream Frosting}
\noindent\rule[0.5ex]{\linewidth}{0.5pt}

%Ingredients
\begin{framed}
    \begin{itemize}
        \item 1 1/2 cup unsalted butter
        \item 2 tablespoons mil
        \item 9 ounces, weight semisweet chocolate, melted and cooled to lukewarm (see note)
        \item 1 teaspoon Vanilla Extract
        \item 1 1/4 cups powdered sugar, sifted
    \end{itemize}
\end{framed}

%Instructions
\begin{enumerate}
    \item 
        In a large mixing bowl, beat the butter using an electric mixer on a medium speed for about 3 minutes or until creamy.
    \item 
        Add the milk carefully and beat until smooth. 
    \item 
        Add the melted chocolate and beat well for 2 minutes. 
    \item 
        Add the vanilla and beat for 3 minutes
    \item 
        Gradually add in the sugar and beat on low speed until creamy and of desired consistency. (Yields enough frosting for a 9-inch 2-layer cake or about 2 dozen cupcakes.
\end{enumerate}

\paragraph
(Note: to melt the chocolate, place it in a double boiler over simmering water on low heat for 5-10 minutes. Stir occasionally until the chocolate is completely smooth and no pieces remain. Remove from heat and let cool 5-15 minutes or until lukewarm.)
\newpage

% -----------------------------------------------------
\subsection{Mary Martins's Chocolate Chip Cookies} 
\noindent\rule[0.5ex]{\linewidth}{1pt}

%Ingredients
\begin{framed}
    \begin{itemize}
        \item 1/2 cup butter
        \item 1/2 cup margarine
        \item 1/3 cup shortening
        \item 1 cup sugar
        \item 1 cup brown sugar
        \item 2 eggs
        \item 2 teaspoons vanilla
        \item 3 3/4 cups flour
        \item 1 teaspoon baking soda
        \item 1 teaspoon salt
        \item 12 ounces chocolate chips
    \end{itemize}
\end{framed}

%Instructions
\begin{enumerate}
    \item 
        Cream together butter, margerine, shortening and sugars (easiest in an electric mixer).
    \item 
        Beat in eggs and vanilla.
    \item 
        Stir together in another dish: flour baking soda and salt - add in to creamed mixture a little at a time.
    \item 
        Stir in chocolate chips. Form into balls about 1 inch thick and place on large baking sheet.
    \item 
        Bake at 375 degrees F for 8-10 minutes (yields 4 dozen).
\end{enumerate}
\newpage

% -----------------------------------------------------
\subsection{Old-Fashioned Pecan Pie} 
\noindent\rule[0.5ex]{\linewidth}{1pt}

%Ingredients
\begin{framed}
    \begin{itemize}
        \item Your favorite 9" single pie crust, prepared
        \item 1/2 cup (1 stick) unsalted butter
        \item 3 tablespoons all-purpose flour
        \item 2 1/8 cups light brown sugar
        \item 1/2 teaspoon salt
        \item 1/4 teaspoon vanilla-butternut flavor, optional
        \item 6 tablespoons milk
        \item 3 large eggs
        \item 2 teaspoons vinegar
        \item 2 teaspoons vanilla extract
        \item 1 1/2 cups pecans, diced and whole
    \end{itemize}
\end{framed}

%Instructions
\begin{enumerate}
    \item 
        Preheat the oven to 375 degrees F. Roll out th epastry and place it in a greased 9" pie plate. Flute the edges decoratively.
    \item 
        Melt the butter and set it aside to cool.
    \item 
        In a large bowl, mix together the flour, sugar, and salt.
    \item 
        Add the milk and eggs and beat well.
    \item 
        Stir in the vinegar, vanilla, and flavoring, if using, then the butter and nuts.
    \item 
        Pour the mixture into the crust and bake for 45 to 50 minutes, until the filling is set most of the way to the center, with a 1 1/2" "puddle" that's still jiggly in the center.
    \item 
        Remove from the oven (the pie will finish seting up as it sits) and cool completely before slicing. (yield: 1 pie, 12 servings)
\end{enumerate}
\newpage 

% -----------------------------------------------------
\subsection{Orange Pecan Crunch [Christmas]} 
\noindent\rule[0.5ex]{\linewidth}{1pt}

%Ingredients
\begin{framed}
    \begin{itemize}
        \item 1 1/2 cups pecan halves
        \item 1/2 cup sugar
        \item 2 tbsp butter
        \item 1/2 tsp vanilla
        \item 1 1/2 tsp finely grated orange peel
    \end{itemize}
\end{framed}

%Instructions
\begin{enumerate}
    \item 
        Line a cookie sheet with foil. Butter it. Set pan aside.
    \item  
        In a heavy skillet combine pecan halves, sugar, butter, and vanilla. Cooke over mediumhigh heat, shaking skillet occasionally. DO NOT STIR.
    \item 
        When sugar begins to melt, reduce heat to low. Cook, stirring frequently til sugar is golden brown. 
    \item 
        Quickly stir in orange peel. Immediately spread onto the prepared baking sheet. Cool. Break into bite sized pieces.
\end{enumerate}
\newpage

% -----------------------------------------------------
\subsection{Pudding \& Cream Cheese Mystery Desert} 
\noindent\rule[0.5ex]{\linewidth}{1pt}

%Ingredients
\begin{framed}
    \begin{itemize}
        \item 2 cup flower 
        \item 1 cup Butter
        \item 1 cup chopped walnuts
        \item 9 oz cool whip
        \item 8 oz cream cream cheese 
        \item 2/3 cup powdered sugar
        \item 2 packages chocolate instant Pudding
        \item cool whip (as a topping)
    \end{itemize}
\end{framed}

%Instructions
\begin{enumerate}
    \item 
        Bake flour, butter, and walnuts at 350 degrees 15minutes on 9x13 inch pan. Let cool
    \item 
        Mix 9 oz cool whip, cream cheese and powdered sugar together. Layer mixture on top of bottom crumble.
    \item 
        Make chocolate pudding according to instructions on package. Layer pudding on top of previous layer in pan.
    \item 
        Layer cool whip as final layer on pan. 
\end{enumerate}
\newpage

% -----------------------------------------------------
\subsection{Rhubarb Pie} 
\noindent\rule[0.5ex]{\linewidth}{1pt}

%Ingredients
\begin{framed}
    \begin{itemize}
        \item 1 cup sugar
        \item 2 tsp cornstarch (I usually add a little more)
        \item 1 small package of raspberry jello
        \item 2 1/2 cups cut rhubarb (extra rhubarb doesn't hurt, it's a two crust pie)
    \end{itemize}
\end{framed}

%Instructions
\begin{enumerate}
    \item 
        Mix sugar, cornstarch, and raspberry jello.
    \item 
        Add mix to rhubarb. 
    \item
        Bake 15 min at 400, reduce to 350 and bake until brown.
\end{enumerate}
\newpage

% -----------------------------------------------------
\subsection{Rice Pudding [Christmas]} 
\noindent\rule[0.5ex]{\linewidth}{1pt}

%Ingredients
\begin{framed}
    \begin{itemize}
        \item 1 cup white rice.
        \item approx. 2 cups water
        \item 1 quart milk 
        \item 1 can evaporated milk 
        \item 1 cup sugar 
        \item 3 beaten eggs 
        \item 1 tsp vanilla extract
        \item Raisins to taste
        \item pinch of cinnamon
    \end{itemize}
\end{framed}

%Instructions
\begin{enumerate}
    \item 
        Cook rice according to packaging directions.
    \item 
        Add milk and evaporated milk to rice in a pot. Bring to a boil stirring often. Add sugar, stir, and boil *slowly* for about 20 minutes stirring constantly. 
    \item 
        Beat eggs in a separate bowl and slowly add a tbsp or two of the rice mixture at a time to the eggs, stirring well after each addition, until you've added about 1 cup of rice mixture to the eggs and both are mixed together well. 
    \item 
        Mix egg/rice mixture into the original pot with the rest of the heated rice mixture. Boil an additional 4 or 5 minutes stirring slowly.
    \item 
        Remove form heat, add vanilla extract and raisins. Mix well. Sprinkle cinnamon on top. Chill for a few hours till quite cool.
\end{enumerate}
\newpage

% -----------------------------------------------------
\subsection{Saltine Cracker Candy [Christmas]} 
\noindent\rule[0.5ex]{\linewidth}{1pt}

%Ingredients
\begin{framed}
    \begin{itemize}
        \item 1 tube of saltines (~ 40 crackers +)
        \item 1 cup (2 sticks) butter (either salted or unsalted; do not use margerine)
        \item 1 cup + 2 tbsp firmly packed brown sugar
        \item 2 cups chocolate chips (I like semi-sweet)
        \item 1/2 to 3/4 cup toasted chopped pecans (or toasted chopped almonds)
    \end{itemize}
\end{framed}

\paragraph
(If using a large cookie sheet, multiply recipe by 1.5)

%Instructions
\begin{enumerate}
    \item 
        Preheat oven to 350 degrees F. Line cookie sheet with foil \& spray with non-stick cooking spray.
    \item 
        Line crackers end to end on sheet, breaking crackers to fit the end if necessary.
    \item 
        In a medium saucepan, melt butter on low heat. Once melted, add brown sugar and turn heat to medium.
    \item 
        Continue stirring until boiling - then boil for 3 minutes (set timer).
    \item 
        Pour over crackers, spread evenly working quickly. 
    \item 
        Put sheet in oven - bake for 5 minutes.
    \item 
        remove -pour chocolate chips ove rtop - place a sheet of foil over and let melt for ~3 minutes.
    \item 
        Spread melted chocolate over the top evenly. Add nuts to top - replace foil and refrigerate until set (can be frozen).
\end{enumerate}
\newpage

% -----------------------------------------------------
\subsection{Shortbread Cookies [Christmas]} 
\noindent\rule[0.5ex]{\linewidth}{1pt}

%Ingredients
\begin{framed}
    \begin{itemize}
        \item 1 lb butter
        \item 3 cups sifted flour
        \item 1/2 cup corn starch
        \item 1 cup sifted powdered sugar
        \item 1 tsp vanilla
    \end{itemize}
\end{framed}

%Instructions
\begin{enumerate}
    \item 
        Cream butter well
    \item
        Add flour, corn starch, sugar and vanilla, beat till mixture is like whipped cream (can take quite a while).
    \item 
        Cook on greased cookie sheet for 12 minutes at 350 degrees F.
\end{enumerate}
\newpage

% -----------------------------------------------------
\subsection{Tom Dennis Master Custard Ice Cream Recipe} 
\noindent\rule[0.5ex]{\linewidth}{1pt}

%Ingredients
\begin{framed}
    \begin{itemize}
        \item 2 cups whole milk
        \item 1 cup sugar
        \item 4 egg yolks
        \item pinch of salt
        \item 2 cups of 1/2\&1/2 (milk and cream base, non-alcoholic)
        \item 2 cups Cream
        \item 3 teaspoons vanilla extract
    \end{itemize}
\end{framed}

%Instructions
\begin{enumerate}
    \item 
        In pan wisk milk, sugar, egg yokes, and salt on medium heat until mixture simmers.
    \item 
        Lower heat, wisk 5 minutes till mixture thickens.
    \item 
        Strain into a bowl and wisk in 1/2\&1/2, cream, and vanilla.
    \item 
        Churn in an ice cream machine according to manufacturer's instructions. Serve directly from machine for soft serve, or store in freezer till needed
\end{enumerate}

%Flavors
\noindent\rule[0.5ex]{\linewidth}{0.5pt}
\paragraph 
(The following flavors can create thicker custard than usual, be aware that manual churning may be required if the ice cream machine is not powerful enough. Alternatively, one can flip the ratio of milk to cream base to the following alternate measurements to thin the recipe as provided in each flavor.)

\subsubsection{Almond Flavor}
\noindent\rule[0.5ex]{\linewidth}{0.5pt}
\begin{framed}
    \begin{itemize}
        \item alternate: 3 cups whole milk
        \item alternate: 1 1/2 cups of 1/2\&1/2 (milk and cream base, non-alcoholic)
        \item alternate: 1 1/2 cups Cream
        \item 1 cup sugar
        \item 4 egg yolks
        \item pinch of salt
        \item 3 teaspoons vanilla extract
        \item 1/2 cup sliced almonds
        \item 1 cup sliced almonds
        \item 2 tablespoons suger
        \item pinch of salt
    \end{itemize}
\end{framed}
\begin{enumerate}
    \item 
        In a medium saucepan over medium heat, cook 1/2 cup almonds with 2 tablespoons of sugar and a pinch of salt until deeply golden and caramelized (approx. 10 minutes). Transfer to a plate and set aside.
    \item 
        In the same pot, toast 1 cup sliced almonds until deeply golden (approx. 5 minutes). Proceed with base recipe in the same pot. Let custart steep off the heat for 1 hour before straining. 
    \item 
        Mix in the sweetened carmalized almonds. Chill.

\end{enumerate}

\subsubsection{Pistachio Flavor}
\noindent\rule[0.5ex]{\linewidth}{0.5pt}
\begin{framed}
    \begin{itemize}
        \item alternate: 4 cups whole milk
        \item alternate: 1 cups of 1/2\&1/2 (milk and cream base, non-alcoholic)
        \item alternate: 1 cups Cream
        \item 1 cup sugar
        \item 4 egg yolks
        \item pinch of salt
        \item 3 teaspoons vanilla extract
        \item 1 cup pistachio paste 
        \item 1/4 teaspoon almond extract
    \end{itemize}
\end{framed}
\begin{enumerate}
    \item 
        Make the base ice cream. Whisk pistachio paste and almond extract into warm straned base. Chill
\end{enumerate}

\subsubsection{Peanut Butter Flavor}
\noindent\rule[0.5ex]{\linewidth}{0.5pt}
\begin{framed}
    \begin{itemize} 
        \item alternate: 4 cups whole milk
        \item alternate: 1 cups of 1/2\&1/2 (milk and cream base, non-alcoholic)
        \item alternate: 1 cups Cream
        \item 1 cup sugar
        \item 4 egg yolks
        \item pinch of salt
        \item 3 teaspoons vanilla extract
        \item 1 cup natural smooth peanut butter 
        \item 1/2 teaspoon vanilla extract
    \end{itemize}
\end{framed}
\begin{enumerate}
    \item 
        Make the base ice cream. Whicks peanut butter and 1/2 teaspoon vanilla extract into warm, strained base. Chill.
\end{enumerate}

\subsubsection{Coconut Flavor}
\noindent\rule[0.5ex]{\linewidth}{0.5pt}
\begin{framed}
    \begin{itemize}
        \item alternate: 2 cups whole milk
        \item alternate: 1 cups of 1/2\&1/2 (milk and cream base, non-alcoholic)
        \item alternate: 1 cups Cream
        \item 1 cup sugar
        \item 4 egg yolks
        \item pinch of salt
        \item 3 teaspoons vanilla extract
        \item 1 cup coconut milk
        \item 1/2 cup sweetened shredded coconut 
        \item 1 cup shredded unsweetened coconut
    \end{itemize}
\end{framed}
\begin{enumerate}
    \item 
        In a medium sacuepan, toast sweetened shredded coconut until deeply golden, about 5 minutes. Tansfer to a plate and set aside.
    \item 
        In the same pot, toast shredded unsweetened coconut until deeply golden, approx. 5 minutes. Proceed with base recipe in the same pot. Let custart steep off the heat for 1 hour befor straining. 
    \item 
        Mix in the cooked sweetened shredded coconut. Chill.

\end{enumerate}
\newpage

% =====================================================
\vspace*{\fill}
\begin{center}
    \section{Drinks}
\end{center}
\vspace*{\fill}
\newpage

% -----------------------------------------------------
\subsection{German Mulled Wine [Christmas]}
\noindent\rule[0.5ex]{\linewidth}{1pt}

%Ingredients
\begin{framed}
    \begin{itemize}
        \item 2 medium lemons
        \item 2 medium oranges
        \item 10 whole cloves
        \item 5 cardamon pods
        \item 1 1/4 cups granulated sugar
        \item 1 1/4 cups water
        \item 2 (3-inch) cinnamon sticks
        \item 2 (750-milliliter) bottles of dry red wine, such as Cabernet Sauvignon or Beajolais Nouveau
        \item 1/2 cup brandy
        \item Cheesecloth
        \item Butcher's twine
    \end{itemize}
\end{framed}

%Instructions
\begin{enumerate}
    \item 
        Using a vegetable peeler, remove the zest from the lemons and oranges in wide strips, avoiding the white pith; place the zest in a large saucepan. Juice the lemons and oranges and add the juice to the pan. Place the cloves and cardamom in a small piece of cheesecloth, tie it tightly with butcher’s twine, and add the bundle to the saucepan.
    \item 
        Add the sugar, water, and cinnamon sticks, place the pan over high heat, and bring to a simmer, stirring to dissolve the sugar. Reduce the heat to low and continue to simmer, stirring occasionally, until the mixture is reduced by about one-third, about 20 minutes.
    \item 
        Add the red wine and brandy, stir to combine, and bring just to a simmer (don’t let it boil). Remove from the heat and remove and discard the spice bundle before serving.
\end{enumerate}
\newpage

% -----------------------------------------------------
\subsection{Nicole's Hard Eggnog} 
\noindent\rule[0.5ex]{\linewidth}{1pt}

%Ingredients
\begin{framed}
    \begin{itemize}
        \item 5 eggs (yolks separated from the whites)
        \item 1/3 cup sugar
        \item 2 tsp sugar
        \item 1 cup heavy cream
        \item 2 cups milk (whole or 2\%)
        \item Whisky/Burbon/etc to taste (approx. 2 shots)
        \item Nutmeg and cinnamon to taste
    \end{itemize}
\end{framed}

%Instructions
\begin{enumerate}
    \item 
        In a large bowl whisk the egg yolks until the color becomes lighter.
    \item 
        Add 1/3 cup sugar slowly. Blend together until sugar is disolved. 
    \item 
        Add Cream, Milk, Alcohol and stir together. Add Nutmeg and Cinnamon to taste.  
    \item 
        In a second bowl whisk egg whites with 2 tsp of sugar until peaks form. 
    \item 
        Stir/Fold Egg whites into the first bowl. 
    \item 
        Chill and serve. 
\end{enumerate}
\newpage

% -----------------------------------------------------
\subsection{Old Fashioned Cocktail} 
\noindent\rule[0.5ex]{\linewidth}{1pt}

%Ingredients
\begin{framed}
    \begin{itemize}
        \item 4 dashes Angostura bitters
        \item 1 tsp sugar
        \item 1 orange wheel
        \item 1 maraschino cherry
        \item 1 splash club soda
        \item 2 oz bourbon
    \end{itemize}
    \paragraph 
    (Garnish)
    \begin{itemize}
        \item 1 fresh orange wheel
        \item 1 cherry
    \end{itemize}
\end{framed}

%Instructions
\begin{enumerate}
    \item In an old fashioned drinking glass, muddle the bitters, sugar, orange wheel, cherry and a splash of soda.
    \item Remove the orange rind, add the bourbon and fill with ice.
    \item Garnish with a fresh orange wheel and cherry.
\end{enumerate}
\newpage

\end{document}

